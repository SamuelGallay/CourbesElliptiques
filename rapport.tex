\documentclass{article}

\usepackage[utf8]{inputenc}    % Encodage d'entrée
\usepackage[T1]{fontenc}       % Sinon Babel n'est pas content 
\usepackage{natbib}            % Pour une liste de références plus française
\usepackage{hyperref}          % Pour des liens vers les références
\usepackage{geometry}          % Pour régler les marges
\usepackage{amsfonts, amsmath, amsthm, amssymb} % Pour des symboles de maths
\usepackage{pgfplots}          % Pour les figures
\usepackage{mathtools}         % Des fractions dans des accolades  
\usepackage{setspace}
\usepackage[french]{babel}     % Pour le support de la langue française

\onehalfspacing

% Couleur des liens dans le document
\hypersetup{
  colorlinks = true,
  breaklinks,
  citecolor = [rgb]{.12,.54,.11},
  linkcolor = {blue},
  urlcolor = {blue},
}

% pgfplots version
\pgfplotsset{compat=1.18}
\usepgfplotslibrary{external}
\tikzexternalize

% Pour les marges
\geometry{verbose,tmargin=1in,bmargin=1in,lmargin=0.7in,rmargin=1.1in}

% Théorèmes
\newtheorem{thm}{Théorème}
\newtheorem{prop}{Proposition}
\newtheorem{deff}{Définition}
\newtheorem{lem}{Lemme}
\newtheorem{cor}{Corollaire}

\newcommand{\Fp}{\mathbb{F}_{p}}
\newcommand{\Fq}{\mathbb{F}_{p^2}}
\newcommand{\F}{\mathbb{F}}
\newcommand{\Z}{\mathbb{Z}}

% Pour les sous-sections dans le sommaire
\setcounter{tocdepth}{2}

\title{Courbes Elliptiques}
\author{Samuel \textsc{Gallay}}
\date{\today}


\begin{document}

\maketitle
%\tableofcontents


%\clearpage

\section{Construction d'une courbe initiale}

L'objectif de cette partie est de construire une courbe elliptique supersingulière adaptée. 
On souhaite que cette courbe ait pour cardinalité un entier possédant de petit facteurs premiers pour pouvoir calculer efficacement des isogénies. 
On souhaite aussi pouvoir trouver des courbes de cardinalité arbitrairement grande pour pouvoir adapter la longueur des clefs, et donc la sécurité, du protocole cryptographique.
Puisque toutes les courbes supersingulières sont définies sur $\Fp$ ou $\Fq$, on ne considérera que des courbes à coefficients dans $\Fq$ par la suite.
Plus précisément, nous allons étudier la courbe $E_0 : y^2 = x^3 + b$ sur $\Fq$ avec $b\in\Fp^\times$ et montrer que sous certaines conditions sur $p$, cette courbe est supersingulière et de bonne cardinalité.


Commençons par compter le nombre de points de $E_0$ sur $\Fp$. Pour cela nous montrons d'abord un résultat sur le nombre de carrés dans $\Fp$.

\begin{prop}
  Soit $p$ un nombre premier impair. $\Fp$ contient $\frac{p+1}{2}$ carrés.
\end{prop}

\begin{proof}
  $\Fp^\times$ est un groupe cyclique d'ordre $p-1$. 
  Notons $g$ un générateur de $\Fp^\times$. 
  
  L'application $ \varphi \colon \frac{\Z}{(p-1)\Z} \to \Fp^\times$,
  $[k] \mapsto g^k$
  est bien définie et est un isomorphisme de groupes.
  Soit $a\in \Fp^\times$, et soit $[k] = \psi^{-1}(a)$.
  Si $k$ est pair, ce qui a un sens puisque $p-1$ est pair, $a = g^k = (g^\frac{k}{2})^2$, donc $a$ est un carré.
  Réciproquement, si $a$ est un carré, il existe $[l]\in \Z/(p-1)\Z$ tel que $a = (g^l)^2 = g^{2l}$.
  Alors $[k] = [2l]$ donc $k$ est pair.
  Il y a donc autant de carrés dans $\Fp^\times$ que de nombres pairs dans $\frac{\Z}{(p-1)\Z}$, soit $\frac{p-1}{2}$. 
  En comptant $0$, il y a $\frac{p+1}{2}$ carrés dans $\Fp$. 
\end{proof}

\begin{prop}
  Si $p$ est impair et si $p \equiv 2 [3]$, alors $\#E_0(\Fp) = p+1$.
\end{prop}

\begin{proof}
  Considérons le morphisme de groupes $\varphi \colon \Fp^\times \to \Fp^\times$, $x \mapsto x^3$. 
  Soit $x\in \ker \varphi$, $x^3 = 1$, donc l'ordre de $x$ noté $o(x)$ divise $3$. 
  Par le théorème de Lagrange, $o(x)\ |\ \#\Fp^\times$, donc $o(x)\ |\ p-1$. 
  Or par hypothèse $p-1\equiv 1[3]$, donc $o(x) = 1$, donc $x = 1$.
  Ainsi $\ker \varphi = {1}$, donc $\varphi$ est injective, et donc bijective par égalité des cardinaux.

  On en déduit immédiatement que l'application $x \mapsto x^3 + b$ de $\Fp$ dans $\Fp$ est bijective.
  Nous pouvons maintenant compter le nombre de points de $E$ dans $\Fp$. 
  Si $x^3 + b$ est un carré non nul de $\Fp$, deux valeurs différentes de $y$ sont solutions de l'équation $E$. 
  Si $x^3 + b = 0$, $y = 0$ est la seule valeur qui convient.
  Enfin si $x^3 + b$ n'est pas un carré de $\Fp$, $E$ ne possède pas de solution.
  En n'oubliant pas que le point à l'infini $O$ est aussi solution de $E$, et en utilisant le lemme, on obtient : $\#E(\Fp) = 2\frac{p-1}{2} + 1 + 1 = p+1$.
\end{proof}

À partir de maintenant on suppose que $p$ est un nombre premier impair congru à $2$ modulo $3$. Ainsi $\#E_0(\Fp) = p+1$. Le théorème suivant va nous permettre de calculer $\#E_0(\F_{p^2})$. 

\begin{thm}
  Soit $E/\F_q$ une courbe elliptique et posons $a = q + 1 - \#E(\F_q)$.
  \begin{itemize}
    \item Le morphisme de Frobenius $\phi \colon E \to E$, $(x, y) \mapsto (x^q, y^q)$ vérifie $\phi^2 - a\phi + q = 0$.
    \item Les deux racines complexes $\alpha$ et $\beta$ de $X^2 - aX + q$ vérifient $\#E(\F_{q^n}) = q^n + 1 - \alpha^n - \beta^n$ pour tout $n\ge1$. 
  \end{itemize}
\end{thm}

\begin{proof}
  La preuve de ce théorème est probablement due à Weil. Voir le théorème V.2.3.1 dans le Silverman.
\end{proof}

Le théorème précédent permet de calculer le nombre de points de la courbe $E_0$ sur $\Fq$ en connaissant le nombre de points de la courbe sur $\Fp$.

\begin{cor}
$\#E(\Fq) = (p+1)^2$.
\end{cor}

\begin{proof}
  En utilisant la proposition 2, $a = 0$. $\alpha$ et $\beta$ sont les racines de $X^2 + p$, ainsi $\alpha^2 = \beta^2 = -p$. Donc $\#E(\Fq) = p^2 + 1 - \alpha^2 - \beta^2 = p^2 + 1 + 2p = (p+1)^2$.
\end{proof}


Montrons maintenant que la courbe $E_0$ est super singulière.

\begin{thm}
  Soit $K$ un corps de caractéristique $p\ge 2$, et soit $E/K$ une courbe elliptique. $\mathrm{End}(E)$ est isomorphe soit à un ordre d'un corps quadratique, soit à un ordre dans une algèbre de quaternions. 
\end{thm}

\begin{proof}
  Voir le théorème V.3.1 dans le Silverman.
\end{proof}

\begin{deff}
  Dans le premier cas on dit que $E$ est \textit{ordinaire} et dans le second cas qu'elle est \textit{supersingulière}.
\end{deff}

\begin{cor}
  $E_0$ est supersingulière.
\end{cor}

\begin{proof}
Ainsi, pour prouver qu'une courbe elliptique est supersingulière, il suffit de trouver deux endomorphismes qui ne commutent pas.
Considérons le morphisme de Frobenius $\pi : E_0/\Fq \to E_0^{(p)}/\Fq$, $(x, y) \mapsto (x^p, y^p)$. 
Puisque $E_0$ est définie sur $\Fp$, $\pi$ est un endomorphisme de $E/\Fq$.

Soit $j$ une racine de $X^2 + X + 1$. Par définition, $j$ appartient à $\Fq$. Dans quelles conditions est-ce que $j$ appartient à $\Fp$ ? $j^3 = -j^2 -j = 1$, donc $j$ est d'ordre $3$. Si $j\in \Fp$ alors $3\ |\ p-1$ par le théorème de Lagrange, soit $p\equiv 1[3]$. Par contraposition, si $p \equiv 2 [3]$ alors $j\in\Fq\setminus\Fp$.

Considérons l'endomorphisme $\psi : E_0 \to E_0$, $(x, y) \mapsto (jx, y)$. 
Remarquons que puisque $j^3 = 1$, $\psi$ est bien défini.
De plus $\pi(\psi(x, y)) = \pi(jx, y) = (x^p, y^p)$, mais $\psi(\pi(x, y)) = \psi(x^p, y^p) = (jx^p, y^p)$. 
Puisque $j\neq 1$, $\pi \circ \psi \neq \psi \circ \pi$, donc End($E$) n'est pas commutatif.
Ainsi $E$ est supersingulière.
\end{proof}





\end{document}

