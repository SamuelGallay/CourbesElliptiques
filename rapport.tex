\documentclass{article}

\usepackage[utf8]{inputenc}    % Encodage d'entrée
\usepackage[T1]{fontenc}       % Sinon Babel n'est pas content 
\usepackage{natbib}            % Pour une liste de références plus française
\usepackage{hyperref}          % Pour des liens vers les références
\usepackage{geometry}          % Pour régler les marges
\usepackage{amsfonts, amsmath, amsthm, amssymb} % Pour des symboles de maths
\usepackage{pgfplots}          % Pour les figures
\usepackage{mathtools}         % Des fractions dans des accolades  
\usepackage{setspace}
\usepackage[french]{babel}     % Pour le support de la langue française

\onehalfspacing

% Couleur des liens dans le document
\hypersetup{
  colorlinks = true,
  breaklinks,
  citecolor = [rgb]{.12,.54,.11},
  linkcolor = {blue},
  urlcolor = {blue},
}

% pgfplots version
\pgfplotsset{compat=1.18}
\usepgfplotslibrary{external}
\tikzexternalize

% Pour les marges
\geometry{verbose,tmargin=1in,bmargin=1in,lmargin=0.7in,rmargin=1.1in}

% Théorèmes
\newtheorem{thm}{Théorème}
\newtheorem{prop}{Proposition}
\newtheorem{deff}{Définition}
\newtheorem{lem}{Lemme}

\newcommand{\Fp}{\mathbb{F}_{p}}
\newcommand{\Fq}{\mathbb{F}_{p^2}}
\newcommand{\Z}{\mathbb{Z}}

% Pour les sous-sections dans le sommaire
\setcounter{tocdepth}{2}

\title{Courbes Elliptiques}
\author{Samuel \textsc{Gallay}}
\date{\today}


\begin{document}

\maketitle
%\tableofcontents


%\clearpage

\section{Introduction}

\begin{lem}
  Soit $p$ un nombre premier impair. $\Fp$ contient $\frac{p+1}{2}$ carrés.
\end{lem}

\begin{proof}
  $\Fp^\times$ est un groupe cyclique d'ordre $p-1$. 
  Notons $g$ un générateur de $\Fp^\times$. L'application 
  \begin{align*}
    \varphi \colon \frac{\Z}{(p-1)\Z} &\to \Fp^\times\\
    [k] &\mapsto g^k
  \end{align*}
  est bien définie et est un isomorphisme de groupes.
  Soit $a\in \Fp^\times$, et soit $[k] = \psi^{-1}(a)$.
  Si $k$ est pair, ce qui a un sens puisque $p-1$ est pair, $a = g^k = (g^\frac{k}{2})^2$, donc $a$ est un carré.
  Réciproquement, si $a$ est un carré, il existe $[l]\in \Z/(p-1)\Z$ tel que $a = (g^l)^2 = g^{2l}$.
  Alors $[k] = [2l]$ donc $k$ est pair.

  Il y a donc autant de carrés dans $\Fp^\times$ que de nombres pairs dans $\Z/(p-1)\Z$, soit $\frac{p-1}{2}$. 
  En comptant $0$, il y a $\frac{p+1}{2}$ carrés dans $\Fp$. 
\end{proof}

\begin{prop}
  ($p$ impair)
Soit $E : y^2 = x^3 + b$ une courbe elliptique définie sur $\Fq$ avec $b \in \Fp^{\times}$.
Si $p \equiv 2 [3]$, alors $\#E(\Fq) = (p+1)^2$.
\end{prop}

\begin{proof}
  Commençons par compter le nombre de points de $E$ sur $\Fp$. 
  Considérons le morphisme de groupes $\varphi \colon \Fp^\times \to \Fp^\times$, $x \mapsto x^3$. 
  Soit $x\in \ker \varphi$, $x^3 = 1$, donc l'ordre de $x$ noté $o(x)$ divise $3$. 
  Par le théorème de Lagrange, $o(x)\ |\ \#\Fp^\times$, donc $o(x)\ |\ p-1$. 
  Or par hypothèse $p-1\equiv 1[3]$, donc $o(x) = 1$, donc $x = 1$.
  Ainsi $\ker \varphi = {1}$, donc $\varphi$ est injective, et donc bijective par égalité des cardinaux.

  On en déduit immédiatement que l'application $x \mapsto x^3 + b$ de $\Fp$ dans $\Fp$ est bijective.
  Nous pouvons maintenant compter le nombre de points de $E$ dans $\Fp$. 
  Si $x^3 + b$ est un carré non nul de $\Fp$, deux valeurs différentes de $y$ sont solutions de l'équation $E$. 
  Si $x^3 + b = 0$, $y = 0$ est la seule valeur qui convient.
  Enfin si $x^3 + b$ n'est pas un carré de $\Fp$, $E$ ne possède pas de solution.
  En n'oubliant pas que le point à l'infini $O$ est aussi solution de $E$, et en utilisant le lemme, on obtient : $\#E(\Fp) = 2\frac{p-1}{2} + 1 + 1 = p+1$.
\end{proof}
\end{document}
