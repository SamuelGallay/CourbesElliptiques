\documentclass{article}

\usepackage[utf8]{inputenc}    % Encodage d'entrée
\usepackage[T1]{fontenc}       % Sinon Babel n'est pas content 
\usepackage{natbib}            % Pour une liste de références plus française
\usepackage{hyperref}          % Pour des liens vers les références
\usepackage{geometry}          % Pour régler les marges
\usepackage{amsfonts, amsmath}
\usepackage{amsthm, amssymb} % Pour des symboles de maths
%\usepackage{pgfplots}          % Pour les figures
\usepackage{xcolor}
\usepackage{mathtools}         % Des fractions dans des accolades  
\usepackage{setspace}
\usepackage{tikz}
\usepackage[french]{babel}     % Pour le support de la langue française

\onehalfspacing
% Couleur des liens dans le document
\hypersetup{
  colorlinks = true,
  breaklinks,
  citecolor = [rgb]{.12,.54,.11},
  linkcolor = {black},
  urlcolor = {blue},
}

%\pgfplotsset{compat=1.18}
%\usepgfplotslibrary{external}
%\tikzexternalize
\usetikzlibrary{cd}

% Pour les marges
\geometry{verbose,tmargin=1in,bmargin=1in,lmargin=0.7in,rmargin=1.1in}

% Math Environments
\theoremstyle{plain}% default
\newtheorem{thm}{Théorème}[section] 
\newtheorem{prop}[thm]{Proposition}
\newtheorem{deff}[thm]{Définition}
\newtheorem{lem}[thm]{Lemme}
\newtheorem{cor}[thm]{Corollaire}
\theoremstyle{definition}% default
\newtheorem{ex}[thm]{Exemple}

\newcommand{\Fp}{\mathbb{F}_{p}}
\newcommand{\F}{\mathbb{F}}
\newcommand{\Z}{\mathbb{Z}}
\newcommand{\N}{\mathbb{N}}

\DeclareMathOperator{\End}{End}
\DeclareMathOperator{\tr}{tr}

% Pour les sous-sections dans le sommaire
\setcounter{tocdepth}{2}

\title{Courbes Elliptiques}
\author{Samuel \textsc{Gallay}}
\date{\today}

\begin{document}

\maketitle





\section{Courbes elliptiques}

\subsection{Définitions}

Je donne en dessous deux définitions de ce qu'est une courbe elliptique. La première est plus générale puisqu'elle ne mentionne pas la  caractéristique du corps. La deuxième est plus pratique puisqu'elle permet de stocker une courbe dans un ordinateur, et d'effectuer des calculs dessus. Nous allons montrer que dans un corps de caractéristique différente de $2$ et de $3$, ces deux définitions sont équivalentes.

\begin{deff}[Formelle]
  Une courbe elliptique est une courbe projective lisse de genre $1$ dont a spécifié un point.
\end{deff}

\begin{deff}[Pratique]
  Soit $k$ un corps de caractéristique différente de $2$ et de $3$. Une courbe elliptique sur $k$ est un couple $(a, b)$ d'éléments de $k$ tel que la quantité $\Delta =-16(4a^3+ 27b^2)$ soit non nulle.
\end{deff}

Fixons pour cette partie un corps corps $k$ caractéristique différente de $2$ et de $3$ et notons $\overline{k}$ une de ses clôtures algébriques. Essayons maintenant de comprendre la définition plus formelle. Le premier terme à comprendre est le mot \emph{projectif}.

\begin{deff}
  L'espace projectif $\mathbb{P}^2$ est l'ensemble $$\frac{\overline{k}^3\setminus(0, 0, 0)}{\sim},\text{\ \ où\ \ } (x, y, z) \sim (x', y', z') \iff \exists \lambda \in \overline{k}^\times \text{ tel que } (x, y, z) = (\lambda x', \lambda y', \lambda z')$$
\end{deff}

\begin{deff}
  Une courbe (projective) est une variété de dimension $1$ de l'espace (projectif).
\end{deff}

\begin{deff}
  Un ensemble est dit algébrique s'il est de la forme $$V(I) = \{ P\in \mathbb{P}^2 : f(P) = 0 \text{ pour tout } f \text{ homogène de } I \}$$ pour un certain $I$, où $I$ est un idéal de $\overline{k}[X, Y, Z]$ homogène (c'est à dire engendré par des polynômes homogènes).
\end{deff}

\begin{deff}
  À un ensemble algébrique $V$, on associe un idéal :
  $$I(V) = \{ f\in \overline{k}[X, Y, Z] : f \text{ est homogène et } f(P) = 0 \text{ pour tout } P\in V \}$$
  Si $I(V)$ est premier dans $ \overline{k}[X, Y, Z] $, on dit que $V$ est une variété (projective). Si $I(V)$ est engendré par des polynômes homogènes de $k[X, Y, Z]$, on dit que $V$ est définie sur $k$.  
\end{deff}



\begin{ex}
  Soit $f$ un polynôme de homogène irréductible de $k[X, Y, Z]$. Posons $V = V(\langle f \rangle)$, et montrons que $V$ est une variété. 
  Il est toujours vrai que $I \subseteq I(V(I))$, donc $\langle f \rangle \subseteq I(V)$. 
  Dans ce cas particulier, la réciproque est vraie. Soit $g\in I(V)$ : par la version forte du théorème des zéros de Hilbert, il existe $n\in\N^*$ tel que $g^n\in \langle f \rangle$.
  Puisque $\overline{k}$ est factoriel et puisque $f$ est irréductible,  $\langle f \rangle$ est premier. Ainsi $g\in \langle f \rangle$ et donc $I(V) = \langle f \rangle$.
  On en déduit que $V$ est une variété (projective) définie sur $k$.
\end{ex}

\begin{ex}
  Posons $f = Y^2 Z - X^3 -aXZ^2 -b Z^3$ avec $a$ et $b$ dans $k$. $f$ est irréductible (je ne l'ai pas vraiment écrit en projectif\ldots), donc $V(\langle f \rangle)$ est une variété définie sur $k$.
\end{ex}

On voit apparaître le lien entre les deux définitions. Étant donné un couple d'éléments de $k$, on construit comme ci-dessus une variété projective. La condition sur $\Delta$ implique que cette variété est lisse. Même si je n'ai pas défini la dimension, intuitivement cette variété est de dimension $1$. Enfin, une telle courbe est de genre $1$. Le point spécifié est le point $[0, 1, 0]$. Ceci "justifie" qu'une courbe elliptique au sens de la définition $1$ est bien une courbe elliptique au sens de la définition $1$. 

Ce qui est plus difficile, c'est de montrer que toute courbe elliptique au sens de la première définition est bien une courbe elliptique au sens de la seconde\ldots. L'outil clef est le théorème de Riemann-Roch.

\subsection{Mais pourquoi s'intéresse-t-on aux courbes elliptiques ?}

Il y a une loi de groupe.

\section{Étude du graphe des isogénies}

Le but de cette partie est de montrer que le graphes des isogénies entre $\overline{\F_p}$-isomorphismes de courbes elliptiques supersingulières définies sur $\F_{p^2}$ est connexe. Il me manque encore un bon nombre de définitions et de propriétés : introduire ce qu'est une isogénie, son degré, la notion d'isogénie duale, le pairing de Weil\ldots

\begin{deff}
  Soit $\phi\in \End(E)$, et soit $l$ un nombre premier. $\phi$ induit un endomorphisme $\phi_l : E[l] \to E[l]$. Si $l\neq p$, $\phi_l$ est un endomorphisme du $\Z/l\Z$-espace vectoriel de dimension $2$ $E[l]$. 
\end{deff}

\begin{prop}
  \label{dettr}
  Soit $\phi$ un endomorphisme de $E$, $l$ un nombre premier différent de $p$, et $\phi_p$ l'endomorphisme induit par $\phi$ sur $E[l]$. Alors $\det(\phi_l) = \deg(\phi)$, et $\tr(\phi_l) = 1 + \deg(\phi) - \deg(1-\phi)$, ce qui montre que la trace et le déterminant de $\phi$ peuvent être définis indépendamment du choix de $l$.
\end{prop}

\begin{proof}
  Pairing de Weil.

  Soit $ A = \begin{pmatrix} a & b \\ c & d \end{pmatrix}$ une matrice $2\times 2$. $\det(1-A) = (1-a)(1-d) - bc = 1 -a -b + ad -bc = 1 - \tr(A) + \det(A)$. Ainsi, en utilisant cette égalité et le lien entre le degré et le déterminant d'un endomorphisme on obtient bien $tr(\phi_l) = 1 + deg(\phi) - deg(1-\phi)$.   
\end{proof}

\begin{prop}
  \label{degpi}
  Soit $E$ une courbe elliptique définie sur $\F_q$, et soit $\pi_q$ le morphisme de Frobenius $\pi_q : E \to E^{(q)}, (x, y) \mapsto (x^q, y^q)$. Alors $\pi_q$ est purement inséparable et $\deg(\pi_q) =q$.
\end{prop}

\begin{proof}
  Silverman II.2.11.
\end{proof}

\begin{prop}
  \label{cardpi}
  Avec les mêmes notations que la proposition précédente, le morphisme $1-\pi_q$ est séparable, et $\#E(\F_q) = deg(1 - \pi_q)$.
\end{prop}

\begin{proof}
  Pour la séparabilité de $1-\pi_q$, voir Silverman III.5.5. Pour le cardinal, utiliser III.4.10c.
\end{proof}

\begin{prop}
  Soit $E/\F_q$ une courbe elliptique et $\pi_q$ l'endomorphisme de Frobenius. Posons $t=q+1-\#E(\F_q)$. Alors $X^2 -tX + q = 0$ annule $\pi_q$.
\end{prop}

\begin{proof}
  En utilisant \ref{degpi} et \ref{cardpi}, $t$ se réécrit $t = 1 + \deg(\pi_q) - \deg(1-\pi_q)$. On reconnait l'expression de la trace de $\pi_q$ donnée en \ref{dettr}, et donc $t = \tr(\pi_q)$. Soit $\pi_{q, l}$ l'endomorphisme induit par $\pi_q$ sur $E[l]$. Le polynôme caractéristique de $\phi_l$ est $X^2 -\tr(\pi_{q,l}) X + \det(\pi_{q, l})$, donc $\pi_{q,l}^2 - t\pi_{q,l}+q = 0$. En appliquant une dernière fois \ref{dettr}, on obtient $\deg(\pi_{q}^2 - t\pi_{q}+q ) = \det (\pi_{q,l}^2 - t\pi_{q,l}+q ) = \det(0) = 0$, et donc par définition du degré on obtient l'égalité souhaitée.
\end{proof}

Le résultat ci-dessus est une partie du théorème \ref{weil}.

\begin{deff}
  $E$ est dite supersingulière si $E[p]$ est réduit à $O$. 
\end{deff}

\begin{prop}
  \label{pihatsep}
  Soit $E/\F_q$ une courbe elliptique. La courbe $E$ est supersingulière si et seulement si $\widehat{\pi_q}$ est (purement) inséparable.
\end{prop}

\begin{proof}
   En utilisant le théorème III.6.2 dans le Silverman (il faudrait quand même introduire l'isogénie duale un jour\ldots), on obtient $\widehat{\pi_p} \pi_p = [\deg \pi_p] = [p]$.

  Supposons que $E/\F_q$ soit supersingulière. Ainsi, $\ker (\widehat{\pi_p}\pi_p) = \ker [p] = 0$ (à finir).
\end{proof}

\begin{prop}
  \label{pimodp}
  $E/\F_q$ est supersingulière si et seulement si $\tr(\pi_q) \equiv 0\ [p]$. 
\end{prop}

\begin{proof}
  $[\deg(1-\pi_q)] = (\widehat{[1] - \pi_q})([1]-\pi_q) = [1] - \widehat{\pi_q} - \pi_q + \widehat{\pi_q}\pi_q  $, donc $ \widehat{\pi_q} + \pi_q = [1] + [\deg\pi_q] - [\deg(1-\pi_q)] = [tr(\pi_q)]  $ 
  L'endomorphisme $\pi_q$ est purement inséparable par \ref{degpi}, donc $[\tr(\pi_q)]$ est inséparable si et seulement si $\widehat{\pi_q}$ est inséparable, si et seulement si $E$ est supersingulière par \ref{pihatsep}. De plus, puisque $\deg[\tr \pi_q] =(\tr \pi_q)^2$ et puisque $p$ est premier, $E$ est supersingulière si et seulement si $p$ divise $\tr\pi_q$. 
\end{proof}


\begin{thm}[Hasse]
  \label{hasse}
  Soit $E$ une courbe elliptique définie sur $\F_q$. Posons $t = \tr(\pi_q) = \# E(\F_q) - q -1$. Alors $|t| \le 2\sqrt{q}$.
\end{thm}

\begin{proof}
  Cauchy-Schwarz, voir Silverman V.1.1.  
\end{proof}

\begin{prop}
  Toute courbe elliptique $E/\overline{\F_p}$ a son $j$-invariant dans $\F_{p^2}$, donc une telle courbe est $\overline{\F_p}$-isomorphe à une courbe définie sur $\F_{p^2}$. 
\end{prop}

\begin{prop}
\label{cardsuper} 
  Soit $E/\F_{p^2}$ une courbe supersingulière définie sur $\F_{p^2}$. $E$ est isomorphe à une courbe définie sur $\F_{p^2}$ de cardinal $E(\F_{p^2}) = p^2 + 1 + 2p = (p+1)^2$.
\end{prop}

\begin{proof}
  Soit $\pi_{p^2}$ le Frobenius de la courbe $E$. En combinant la borne de Hasse $-2p \le \tr(\pi_{p^2}) \le 2p$ (théorème \ref{hasse}) et la proposition \ref{pimodp} $\tr(\pi_{p^2}) \equiv 0\ [p]$, on obtient $\tr(\pi_{p^2}) \in \{-2p, -p, 0, p, 2p\}$.  Nous allons distinguer les cas selon la valeur de $\tr(\pi_{p^2})$.
   
  \vspace{1em}
  Si $\tr(\pi_{p^2}) = -2p$, il n'y a rien à faire. On rappelle juste que $E(\F_{p^2}) = p^2 + 1 - \tr(\pi_{p^2})$ 
    
  \vspace{1em}
  Supposons que $\tr(\pi_{p^2}) = 2p$. Tous les $\overline{\F_p}$-isomorphismes entre courbes elliptiques préservant la forme de Weierstrass réduite sont de la forme 
      $\varphi : E \to E', (x, y) \mapsto (u^2 x, u^3 y)$ avec $u\in\overline{\F_p}$. Notons $E : y^2 = x^3 + ax +b$. L'équation $E'$ s'écrit donc $E' : u^6 y^2 = u^6 x^3 + u^2 a x + b$, (soit $E' : y^2 = x^3 + u^{-4} a x + u^{-6}b$ si l'on souhaite une forme réduite). On peut choisir $u^2\in\F_{p^2}$ avec $u\notin \F_{p^2}$, et alors $E'$ est définie sur $\F_{p^2}$, mais ne possède pas le même nombre de points dans $\F_{p^2}$ que $E$.

      Comptons le nombre de points de $\#E(\F_{p^2}) + \#E'(\F_{p^2})$. $E'$ se réécrit encore $E' : u^2 (u^2 y)^2 = (u^2 x)^3 + a (u^2 x) + b$. Puisque $u^2\in \F_{p^2}^\times$, $\#E'(\F_{p^2}) = \#E''(\F_{p^2})$ avec $E'' : u^2 y^2 = x^3 + a x + b$, soit encore $E'' : y^2 = u^{-2} (x^3 + ax + b)$.
      La multiplication par $u^2$ est une bijection entre les carrés de $\F_{p^2}^\times$ et les éléments qui ne sont pas des carrés de $\F_{p^2}^\times$. 
      Ainsi, si $x^3 + ax + b$ est un carré non nul de $\F_{p^2}$, $E$ possède deux solutions pour un tel $x$ et $E'$ aucune. Si ce n'est pas un carré, $E$ ne possède pas de solutions et $E'$ en possède deux. Enfin si cette quantité est nulle, $E$ et $E'$ possèdent chacune une solution. En n'oubliant que le point à l'infini appartient aussi aux deux courbes, on obtient :
      $ \#E(\F_{p^2}) + \#E'(\F_{p^2}) = (p^2 - 1) + (p^2 -1) + 2 + 2 = 2(p^2 + 1)$. Ainsi $\#E'(\F_{p^2}) = p^2 + 1 + tr(\pi_{p^2}) = p^2 + 1 + 2p$. 
      {\color{red} (Il faut que $\F_{p^2} \to \F_{p^2} : x \mapsto x^3 + ax +b$ soit bijective pour que cette démonstration soit correcte, ce qui est vrai si $3$ ne divise pas $p^2 - 1=(p+1)(p-1)$, ça semble pas bien parti... En fait c'est bon, j'ai compris, il faut juste que je le rédige mieux.)}

\vspace{1em}
Pour traiter les derniers cas nous serons amenés à étudier les automorphismes de $E$ (voir le théorème III.10.1 dans Silverman). Puisque $E$ est supersingulière, $\widehat{\pi_p} : E^{(p)} \to E$ est purement inséparable donc se factorise en  $\widehat{\pi_p} = \psi \circ \pi_p$. Sachant que $\deg(\widehat{\pi_p}) = \deg(\pi_p) = p^2$, $\psi$ est de degré $1$ et donc est un isomorphisme. De plus $\widehat{\pi_p}\pi_p = [p]$, et $E^{(p^2)} = E$, le diagramme suivant est commutatif: 

\begin{center}
  \begin{tikzcd}
          E \arrow[rrd, "p"] \arrow[r, "\pi_p"] & E^{(p)} \arrow[rd, "\widehat{\pi_p}"] \arrow[r, "\pi_p"] &E \arrow[d, "\psi"]\\
                                 &  & E
  \end{tikzcd}
\end{center}
Ainsi $\pi_{p^2} = \psi^{-1} \circ [p]$.

\vspace{1em}
Dans le cas où $\tr(\pi_{p^2}) = 0$, $\pi_{p^2}^2 + [p^2] = 0$ donc $-[p^2] = \pi_{p^2}^2 = (\psi^{-1} \circ [p])^2 = \psi^{-2} \circ [p^2]$ puisque tout morphisme commute avec $[m]$. Alors $(\psi^{-2} +1)\circ [p^2] = 0\in \End(E)$ donc $[p^2]$ étant surjective, $\psi^2 = -\text{Id}$.
Ainsi $\psi$ est d'ordre $4$, donc par le théorème III.10.1 (Silverman) (caractéristique $\neq 2, 3$) $j(E) = 1728$.
Ainsi $E$ est $\overline{\F_p}$-isomorphe à une courbe $y^2=x^3 + ax$ qui est de cardinal $(q+1)^2$.

\vspace{1em}
Dans les dernier cas, où $\tr(\pi_{p^2}) = \pm p$, $X^2 \pm p X +p^2$ annule $\pi_{p^2}^2$ et puisque $X^6 - p^6 = p^6 ((\frac{X}{p})^6 - 1) = (X-1)(X+1)(X^2 -pX + p^2)(X^2 + pX + p^2)$, $\pi_{p^2}^6 = [p^6]$. Ainsi $[p^6] = \pi_{p^2}^6  = (\psi^{-1} \circ [p])^6 = \psi^{-6} \circ [p^6]$, donc $\psi^6 = \text{Id}$. Le théorème III.10.1 permet de conclure si l'ordre de $\psi$ est $3$ ou $6$, il suffit donc de montrer que l'ordre de $\psi$ ne peut être ni $1$, ni $2$.
Si $\psi$ est d'ordre $1$, $\pi_{p^2} = [p]$, donc $\tr(\pi_{p^2}) = \tr([p]) = 1 + \deg([p]) - \deg([1-p]) = 1 + p^2 -(p-1)^2 = 2p$, ce qui contredit l'hypothèse $\tr(\pi_{p^2}) = \pm p$. 
Si $\psi$ est d'ordre $2$, $\pi_{p^2}^2 = (\psi^{-1} \circ [p])^2 = \psi^{-2} \circ [p^2] = - [p^2]$, donc $\pi_{p^2}^6 = -[p^6]$ ce qui contredit $[p^6] = \pi_{p^2}^6$. 
\end{proof}

\begin{thm}[Tate]
  \label{tate}
  Étant donné deux courbes elliptiques $E_1$ et $E_2$ (définies sur $\F_{q}$ ou $\overline{\F_p}$ ?), il existe une isogénie (non constante par définition) $\varphi : E_1 \to E_2$ définie sur $\F_q$ si et seulement si $\# E_1(\F_q) = \# E_2(\F_q)$. \end{thm}

\begin{proof}
  Admise.
\end{proof}

\begin{cor}
  Le graphe des isogénies (que l'on a pas encore construit) entre courbes supersingulières définies sur $\F_{p^2}$ est connexe.
\end{cor}

\begin{proof}
  En combinant du théorème de Tate (\ref{tate}) et de la proposition \ref{cardsuper}, on obtient qu'entre deux classe de $\overline{\F_p}$-isomorphisme entre courbes elliptiques supersingulières définies sur $\F_{p^2}$ il existe toujours une isogénie.
\end{proof}

\section{Construction d'une courbe initiale}

L'objectif de cette partie est de construire une courbe elliptique supersingulière adaptée. 
On souhaite que cette courbe ait pour cardinalité un entier possédant de petit facteurs premiers pour pouvoir calculer efficacement des isogénies. 
On souhaite aussi pouvoir trouver des courbes de cardinalité arbitrairement grande pour pouvoir adapter la longueur des clefs, et donc la sécurité, du protocole cryptographique.
Puisque chaque courbe supersingulière est $\overline{\F_p}$-isomorphe à une courbe définie sur $\F_{p}$ ou $\F_{p^2}$, on ne considérera que des courbes à coefficients dans $\F_{p^2}$ par la suite.
Plus précisément, nous allons étudier la courbe $E_0 : y^2 = x^3 + b$ sur $\F_{p^2}$ avec $b\in\F_{p}^\times$ et montrer que sous certaines conditions sur $p$, cette courbe est supersingulière et de bonne cardinalité.


Commençons par compter le nombre de points de $E_0$ sur $\F_{p}$. Pour cela nous montrons d'abord un résultat sur le nombre de carrés dans $\F_{p}$.

\begin{prop}
  Soit $p$ un nombre premier impair. $\F_{p}$ contient $\frac{p+1}{2}$ carrés.
\end{prop}

\begin{proof}
  $\F_{p}^\times$ est un groupe cyclique d'ordre $p-1$. 
  Notons $g$ un générateur de $\F_{p}^\times$. 
  
  L'application $ \varphi \colon \frac{\Z}{(p-1)\Z} \to \F_{p}^\times$,
  $[k] \mapsto g^k$
  est bien définie et est un isomorphisme de groupes.
  Soit $a\in \F_{p}^\times$, et soit $[k] = \psi^{-1}(a)$.
  Si $k$ est pair, ce qui a un sens puisque $p-1$ est pair, $a = g^k = (g^\frac{k}{2})^2$, donc $a$ est un carré.
  Réciproquement, si $a$ est un carré, il existe $[l]\in \Z/(p-1)\Z$ tel que $a = (g^l)^2 = g^{2l}$.
  Alors $[k] = [2l]$ donc $k$ est pair.
  Il y a donc autant de carrés dans $\F_{p}^\times$ que de nombres pairs dans $\Z/{(p-1)\Z}$, soit $\frac{p-1}{2}$. 
  En comptant $0$, il y a $\frac{p+1}{2}$ carrés dans $\F_{p}$. 
\end{proof}

\begin{prop}
  Si $p$ est impair et si $p \equiv 2 [3]$, alors $\#E_0(\F_{p}) = p+1$.
\end{prop}

\begin{proof}
  Considérons le morphisme de groupes $\varphi \colon \F_{p}^\times \to \F_{p}^\times$, $x \mapsto x^3$. 
  Soit $x\in \ker \varphi$, $x^3 = 1$, donc l'ordre de $x$ noté $o(x)$ divise $3$. 
  Par le théorème de Lagrange, $o(x)\ |\ \#\F_{p}^\times$, donc $o(x)\ |\ p-1$. 
  Or par hypothèse $p-1\equiv 1[3]$, donc $o(x) = 1$, donc $x = 1$.
  Ainsi $\ker \varphi = {1}$, donc $\varphi$ est injective, et donc bijective par égalité des cardinaux.

  On en déduit immédiatement que l'application $x \mapsto x^3 + b$ de $\F_{p}$ dans $\F_{p}$ est bijective.
  Nous pouvons maintenant compter le nombre de points de $E$ dans $\F_{p}$. 
  Si $x^3 + b$ est un carré non nul de $\F_{p}$, deux valeurs différentes de $y$ sont solutions de l'équation $E$. 
  Si $x^3 + b = 0$, $y = 0$ est la seule valeur qui convient.
  Enfin si $x^3 + b$ n'est pas un carré de $\F_{p}$, $E$ ne possède pas de solution.
  En n'oubliant pas que le point à l'infini $O$ est aussi solution de $E$, et en utilisant le lemme, on obtient : $\#E(\F_{p}) = 2\frac{p-1}{2} + 1 + 1 = p+1$.
\end{proof}

À partir de maintenant on suppose que $p$ est un nombre premier impair congru à $2$ modulo $3$. 
Ainsi $\#E_0(\F_{p}) = p+1$. 
Le théorème suivant va nous permettre de calculer $\#E_0(\F_{p^2})$. 

\begin{thm}
  \label{weil}
  Soit $E/\F_q$ une courbe elliptique et posons $a = q + 1 - \#E(\F_q)$.
  \begin{itemize}
    \item Le morphisme de Frobenius $\phi \colon E \to E$, $(x, y) \mapsto (x^q, y^q)$ vérifie $\phi^2 - a\phi + q = 0$.
    \item Les deux racines complexes $\alpha$ et $\beta$ de $X^2 - aX + q$ vérifient $\#E(\F_{q^n}) = q^n + 1 - \alpha^n - \beta^n$ pour tout $n\ge1$. 
  \end{itemize}
\end{thm}

\begin{proof}
  La preuve de ce théorème est probablement due à Weil. Voir le théorème V.2.3.1 dans le Silverman.
\end{proof}

Le théorème précédent permet de calculer le nombre de points de la courbe $E_0$ sur $\F_{p^2}$ en connaissant le nombre de points de la courbe sur $\F_{p}$.

\begin{cor}
$\#E_0(\F_{p^2}) = (p+1)^2$.
\end{cor}

\begin{proof}
  En utilisant la proposition 2, $a = 0$. $\alpha$ et $\beta$ sont les racines de $X^2 + p$, ainsi $\alpha^2 = \beta^2 = -p$. Donc $\#E(\F_{p^2}) = p^2 + 1 - \alpha^2 - \beta^2 = p^2 + 1 + 2p = (p+1)^2$.
\end{proof}


Montrons maintenant que la courbe $E_0$ est supersingulière.

\begin{thm}
  Soit $K$ un corps de caractéristique $p\ge 2$, et soit $E/K$ une courbe elliptique. $\mathrm{End}(E)$ est isomorphe soit à un ordre d'un corps quadratique, soit à un ordre dans une algèbre de quaternions. 
\end{thm}

\begin{proof}
  Voir le théorème V.3.1 dans le Silverman.
\end{proof}

\begin{deff}
  Dans le premier cas on dit que $E$ est \textit{ordinaire} et dans le second cas qu'elle est \textit{supersingulière}.
\end{deff}

\begin{cor}
  $E_0$ est supersingulière.
\end{cor}

\begin{proof}
Ainsi, pour prouver qu'une courbe elliptique est supersingulière, il suffit de trouver deux endomorphismes qui ne commutent pas.
Considérons le morphisme de Frobenius $\pi : E_0/\F_{p^2} \to E_0^{(p)}/\F_{p^2}$, $(x, y) \mapsto (x^p, y^p)$. 
Puisque $E_0$ est définie sur $\F_{p}$, $\pi$ est un endomorphisme de $E/\F_{p^2}$.

Soit $j$ une racine de $X^2 + X + 1$. 
Par définition, $j$ appartient à $\F_{p^2}$. 
Dans quelles conditions est-ce que $j$ appartient à $\F_{p}$ ? 
$j^3 = -j^2 -j = 1$, donc $j$ est d'ordre $3$. 
Si $j\in \F_{p}$ alors $3\ |\ p-1$ par le théorème de Lagrange, soit $p\equiv 1[3]$. 
Par contraposition, si $p \equiv 2 [3]$ alors $j\in\F_{p^2}\setminus\F_{p}$.

Considérons l'endomorphisme $\psi : E_0 \to E_0$, $(x, y) \mapsto (jx, y)$. 
Remarquons que puisque $j^3 = 1$, $\psi$ est bien défini.
De plus $\pi(\psi(x, y)) = \pi(jx, y) = (j^px^p, y^p)$, mais $\psi(\pi(x, y)) = \psi(x^p, y^p) = (jx^p, y^p)$. 
Puisque $j^p\neq j$, $\pi \circ \psi \neq \psi \circ \pi$, donc End($E$) n'est pas commutatif.
Ainsi $E$ est supersingulière.
\end{proof}

On s'intéresse maintenant à la structure des groupes de torsion de $E_0$ sur $\F_{p^2}$. Pour cela nous avons d'abord besoin d'informations sur la structure des groupes de torsion dans la clôture algébrique de $\F_{p}$.
\begin{deff}
  Soit $E/K$ une courbe elliptique. Pour $m\in\Z$, on définit la $m$-torsion de $E$ par $ E[m] = \left\{ P\in E/\overline{K} \ \vert\  mP = O \right\}$.
\end{deff}

\begin{thm}
  Soit $m\neq0$. Si $char(K)\neq 0$ ou si char($K$) ne divise pas $m$, $E[m] \simeq {\Z}/{m\Z} \times  {\Z}/{m\Z}$
\end{thm}

\begin{proof}
  Voir le corollaire III.6.4 dans le Silverman.
\end{proof}

Le théorème précédent nous permet d'identifier la structure du groupe abélien $E_0(\F_{p^2})$.

\begin{prop}
  $$E_0(\F_{p^2}) \simeq \Z/(p+1)\Z \times \Z/(p+1)\Z$$
\end{prop}
\begin{proof}
  Utilisons à nouveau le théorème \ref{weil} en posant $q = p^2$. 
  Posons $\phi : E(\overline{\F_p}) \to E(\overline{\F_p})$, $\left( x, y \right) \mapsto \left( x^{q}, y^{q} \right)$.
  En constatant que $a=p^2 + 1 - \#E_0(\F_{p^2}) = p^2 + 1 - \left( p + 1 \right) ^2 = -2p$, on obtient que $\phi^2 + 2p\phi + [p^2] = 0$, donc que $\left( \phi + [p] \right) ^2 = 0$ dans End($E(\overline{\F_p})$). 
  Par un argument de degré, cela implique que $\phi = -[p]$. 

  Soit $P\in E(\overline(\F_{p})$. $P \in E(\F_{p^2}) \iff \phi(P) = P$, donc $E(\F_{p^2}) = \ker(\text{Id} - \phi) = \ker([1] + [p]) = \ker([p+1])$.
  En utilisant la proposition précédente, puisque $p$ ne divise pas $p+1$, on obtient le résultat souhaité.
\end{proof}

On considère maintenant la décomposition de $p + 1$ suivante : $p + 1 = 2^a 3^b f$, avec $2 \nmid f$ et $3 \nmid f$. Puisque l'on a déjà supposé $p\ge 3$ et $p \equiv 2\ [3]$, on a $a, b\ge 1$. Avec ces notations, on s'intéresse à la $2^a$ et à la $3^b$-torsion de $E_0$. Pour éviter d'écrire deux fois la même chose, le couple $\left( l, e \right)$ désigne soit le couple $(2, a)$ soit le couple $(3, b)$ dans les preuves suivantes. 

\begin{prop}
  $$E_0(\F_{p^2})[l^e] \simeq \Z/l^e\Z \times \Z/l^e\Z$$
\end{prop}

\begin{proof}
  D'abord, par la proposition précédente, $\Z/l^e\Z \times \Z/l^e\Z$ est isomorphe à un unique sous-groupe de $E_0(\F_{p^2})$.
  Il suffit donc de montrer une double inclusion entre deux sous-groupes de $E_0(\F_{p^2})$.
  L'inclusion \allowbreak$E_0(\F_{p^2})[l^e] \supseteq \Z/l^e\Z \times \Z/l^e\Z$ est évidente en remarquant que tout élément de $\Z/l^e\Z \times \Z/l^e\Z$ a un ordre qui divise $l^e$.
  
  Pour l'inclusion réciproque on peut par exemple utiliser les théorèmes de Sylow. 
  Par définition, $E_0(\F_{p^2})[l^e]$ est un $l$-sous-groupe de $E_0(\F_{p^2})$, il est donc contenu dans un $l$-Sylow. 
  Puisque $E_0(\F_{p^2})$ est abélien, et puisque les $l$-Sylow sont conjugués, il existe un unique $l$-Sylow. 
  En remarquant que $\Z/l^e\Z \times \Z/l^e\Z$ est un $l$-Sylow par la proposition précédente, on démontre l'inclusion manquante et donc l'isomorphisme annoncé. 
\end{proof}

On s'intéresse maintenant au nombre de sous-groupes cycliques de la $2^a$ ou $3^b$-torsion.

\begin{prop}
  $E_0(\F_{p^2})[l^e]$ contient $l^{e-1}(l+1)$ sous-groupes cycliques d'ordre $l^e$.
\end{prop}
\begin{proof}
  Comptons le nombre d'éléments d'ordre $l^e$ dans $\Z/l^e\Z \times \Z/l^e\Z$.
  Soit $\left( x, y \right) \in \Z/l^e\Z \times \Z/l^e\Z$, $o((x, y)) =\textrm{ppcm}(o(x), o(y)) = \max(o(x),o(y))$, donc un couple d'éléments de $\Z/l^e\Z$ est d'ordre $l^e$ si et seulement si un des éléments est d'ordre $l^e$. Alors, 
  $$\#\{(x, y)\ |\  o((x, y)) = l^e\} = \#\{(x, y)\ |\ o(x) = l^e\} +  \#\{(x, y)\ |\ o(y) = l^e\} - \#\{(x, y)\ |\ o(x) = o(y) = l^e\}$$
  Notons $\alpha = l^{e} - l^{e-1}$ le nombre de générateurs de $\Z/l^e\Z$.  
  Ainsi le nombre de sous sous-groupes cycliques d'ordre $l^e$ de la $l^e$-torsion est $({2l^e\alpha -\alpha^2})/{\alpha} = 2l^e-\alpha = l^e + l^{e-1} = l^{e-1}(l + 1)$.
\end{proof}

\begin{prop}
  Soient $P$ et $Q$ deux éléments de $\Z/l^e\Z \times \Z/l^e\Z$. 
  $(P, Q)$ est $\Z/l^e\Z $-libre ssi $(l^{e-1}P, l^{e-1}Q)$ est $\Z/l\Z $-libre. 
\end{prop}

\begin{proof}
  L'équivalence précédente se réécrit :
  $$\left( \forall a, b \in \Z,\ aP + bQ = 0 \Rightarrow a \equiv b \equiv 0\ [l^e] \right) \iff \left( \forall a, b \in \Z,\ a l^{e-1}P + b l^{e-1}Q = 0 \Rightarrow a \equiv b \equiv 0\ [l] \right) $$
  Montrons l'implication directe. Soient $a, b\in \Z$ tels que $a l^{e-1}P + b l^{e-1}Q = 0$. 
  Par hypothèse, $l^{e-1}a \equiv l^{e-1}b \equiv 0\ [l^e]$, ce qui montre directement que $a \equiv b \equiv 0\ [l]$. 

  Montrons maintenant l'implication réciproque. Soient $a, b \in \Z$ tels que $aP + bQ = 0$. 
  Dans le cas où $a$ et $b$ sont non nuls, on réécrit $a = a' l^{v_l(a)}$ et $b = b' l^{v_l(b)}$ avec $v_l(a)$ et $v_l(b)$ les valuations $l$-adiques de $a$ et de $b$. 
  Ainsi, $a' l^{v_l(a)}P + b' l^{v_l(b)}Q = 0$. 
  Sans perte de généralité, on peut supposer que $v_l(a) \le v_l(b)$. 
  Si $v_l(a) \le e-1$, en multipliant l'équation précédente par $l^{e-1-v_l(a)}$ on obtient $a' l^{e-1}P + b' l^{e-1 + v_l(b) - v_l(a)}Q = 0$, et ainsi $a' \equiv 0\ [l]$, ce qui contredit la définition de $v_l(a)$. 
  Ainsi $v_l(b) \ge v_l(a) \ge e$, donc $a \equiv b \equiv 0\ [l^e]$. Le cas $a=b=0$ est trivial. Dans le cas où $a\neq0$ et $b=0$ par exemple, la preuve précédente reste correcte en écrivant $b = 0\cdot l^{v_l(a)}$.
\end{proof}





\end{document}

